\documentclass[a4paper]{article}
\usepackage[mac]{inputenc}
\usepackage{xcolor}
\usepackage{graphicx}
\usepackage{amssymb}
\usepackage{a4wide}
%\usepackage[pdftex,colorlinks=true,pdfstartview=FitV,linkcolor=blue,citecolor=blue,urlcolor=blue]{hyperref}
\usepackage[pdftex,colorlinks=true,pdfstartview=FitV,linkcolor=black,citecolor= black,urlcolor= black]{hyperref}
% \usepackage[pdftex]{hyperref}
%%%%%%%%%%%%%%%%%%%%%%%%%%%%%%%%%%%%%%%%%%%%%%%%%%%%%%%%%%%%
\usepackage{ifthen}
\newboolean{longversion}
\setboolean{longversion}{false} % toggle for long/short version
% \setboolean{longversion}{true} % toggle for long/short version
\ifthenelse{\boolean{longversion}}
{
	\newcommand{\longcv}[1]{#1}
	\newcommand{\shortcv}[1]{}
}{
	\newcommand{\longcv}[1]{}
	\newcommand{\shortcv}[1]{#1}
}
\newcommand{\period}{(since 2007)}
\renewcommand{\labelitemi}{---}% {$\bullet$}

\newcommand{\nbc}[3]{
 {\colorbox{#3}{\bfseries\sffamily\scriptsize\textcolor{white}{#1}}}
 {\textcolor{#3}{\sf\small$\blacktriangleright$\textit{#2}$\blacktriangleleft$}}}
\newcommand\ml[1]{\nbc{ML}{#1}{violet}} % add more author macros here

%%%%%%%%%%%%%%%%%%%%%%%%%%%%%%%%%%%%%%%%%%%%%%%%%%%%%%%%%%%%
\pdfinfo{
  /Title(Curriculum Vitae)
  /Author(Mircea Lungu)
  /Subject()
  /Keywords($Id$)
}
%%%%%%%%%%%%%%%%%%%%%%%%%%%%%%%%%%%%%%%%%%%%%%%%%%%%%%%%%%%%
\begin{document}
\longcv{\title{\textsf{Dr. Mircea F. Lungu \\Curriculum Vit\ae}}}
\shortcv{\title{\textsf{Dr. Mircea F. Lungu \\Curriculum Vit\ae ~(short version)}}}
\pagestyle{myheadings}
\thispagestyle{plain}
\markright{Dr. Mircea F. Lungu CV --- \today}
\shortcv{\date{}}
\maketitle
%%%%%%%%%%%%%%%%%%%%%%%%%%%%%%%%%%%%%%%%%%%%%%%%%%%%%%%%%%%%
\shortcv{\vspace{-2cm}}
\longcv{\hfill\includegraphics[width=3cm]{on-feb-2011-small.jpg}\vspace{-4cm}}

\section{Personal Data}
\begin{tabular}{p{3cm}l}
Name				& Mircea Filip Lungu \\
Residence		& Engestrasse 1, 3012, Bern\\
Mobile			& +41 78 904 3235\\
Work				& Institut f�r Informatik (IAM), Universit�t Bern \\
				& Neubr�ckstrasse 10, CH-3012 Berne, Switzerland \\
Telephone		& +41 31 631 7637 \\
Fax			& +41 31 631 3355 \\
E-mail			& \href{mailto:lungu@iam.unibe.ch}{lungu@iam.unibe.ch} \\
Home page		& \url{http://scg.unibe.ch/staff/mircea} \\
Birthdate		& October 24, 1980 (Arad, Romania)\\
Citizenship		& Romanian \\
\longcv{
Civil status		& Unmarried\\
Hobbies			& Running, swimming, biking, skiing.\\
Languages		& Romanian (mother tongue).\\
				& Fluent in English, French, Italian.\\
				& Basic conversation skills in German.
} % end longcv

\end{tabular}
%%%%%%%%%%%%%%%%%%%%%%%%%%%%%%%%%%%%%%%%%%%%%%%%%%%%%%%%%%%%
\section{Employment History}
\begin{tabular}{p{3.5cm}l}
2010-present		& \emph{Postdoctoral Researcher} \\
				& Institut f�r Informatik, Universit�t Bern \\
2009-2010		& \emph{Postdoctoral Researcher} \\
				& Faculty of Informatics, University of Lugano\\
2004-2009		& \emph{PhD Student} \\
				& Faculty of Informatics, University of Lugano\\
2007			& \emph{Intern/Co-op Researcher} \\
				& IBM TJ Watson Research Center, NY\\
2001-2003		& \emph{Software Developer} \\
				& Computervoice Systems, Romania\\
1999-2001		& \emph{Instructor for the Programming Course} \\
				& Logos Professional School, Timisoara, Romania\\


\longcv{
}
\end{tabular}
% ==========================================================
\subsection*{University education}
\begin{tabular}{p{3cm}p{12cm}}
Ph.D.	& 2009, Faculty of Informatics, University of Lugano\\
		& Thesis: \emph{Reverse Engineering Software Ecosystems} \\
		& Supervisor: Prof. M. Lanza\\
Dipl.Ing.	& 2004, Department of Computer Science, Polytechnic University of Timisoara\\
		& Thesis: \emph{Conformity Strategies - Measures of Software Design Rules}\\
		& Supervisor: Prof. R. Marinescu \\
\end{tabular}
% ==========================================================
\longcv{
\subsection*{Research interests}
Software evolution.
Reverse and re-engineering.
Context-oriented programming.
}

%%%%%%%%%%%%%%%%%%%%%%%%%%%%%%%%%%%%%%%%%%%%%%%%%%%%%%%%%%%%
%\longcv{
\longcv{\section{Professional Activities}}
\shortcv{\section{Selected Professional Activities \period}}

\shortcv{
%:- Reviewer:
\paragraph{Journal Reviewer:}
Journal of Software Maintenance and Evolution,
Empirical Software Engineering,
Elsevier Journal of Systems and Software,
Elsevier Science of Computer Programming
}

\longcv{
%:- Reviewer:
%\paragraph{Project Reviewer.}
%Expert for the CORDIS FP7 Programme
}

%:- Member:
%\paragraph{Committee Member}
%ESEC, Steering Committee (1999-present);

%:- Organizer, co-organizer:
\paragraph{Organizer, Program Chair:}
FAMOOSR 2010 (Organizer), WCRE 2011 (Tool Demo Track Chair), CSMR 2012 (Tool Demo Track Chair)

%:- PC Member:
\paragraph{PC Member:}
ICSE 2012 (Posters \& TD Track), 
WCRE 2012, %\footnote {International Conference on Reverse Engineering}, 
ICSM 2012 (TD Track), 
ICSM 2011 (TD Track), 
WASDETT 2011, 
ESEC/FSE 2011 (TD Track), 
CSMR 2011, 
IWPSE/EVOL 2011, 
WSE 2011, 
IWSECO 2010, 
WSE 2010.
\paragraph{Board Member:}
CHOOSE - Swiss Group for Object Oriented Systems and Environments \footnote{\url{http://choose.s-i.ch}} 2011 - present.


%%%%%%%%%%%%%%%%%%%%%%%%%%%%%%%%%%%%%%%%%%%%%%%%%%%%%%%%%%%%
\section{Scholarships, honours and awards}
\begin{tabular}{p{1cm}p{12cm}}

2010 & Best Paper Award with Lile Hattori at IWPSE-EVOL 2010 for {\em Replaying Past Changes in Multi-Developer Projects}\\

2007 & 1st Place at the ESUG Innovation Awards for the {\em Small Project Observatory} a platform for the visualization, monitoring, and analysis of software ecosystems\\

2006 & Best Poster Award at the 3rd International ACM Symposium on Software Visualization, Brighton 2006, for the poster {\em Cutting Edge Software Visualization} \\

2003 & {\em Best software engineer} award in the LOOSE Software Engineering Lab at the Polytechnic University of Timisoara, Romania \\

2002 & 2nd Place with the Polytechnica Timisoara's team at the International {\em Hard \& Soft} Contest, Suceava, Romania. The contest was based on Image Processing and the project we developed was a security sistem \\

% 1999 & 3rd Place at the National Student Software Development Contest, Focsani, Romania \\
%
\end{tabular}

%%%%%%%%%%%%%%%%%%%%%%%%%%%%%%%%%%%%%%%%%%%%%%%%%%%%%%%%%%%%
\section{Invited Talks \period}

\begin{itemize}
\item [---] Invited Talk at University of California, Irvine. {\em Reverse Engineering Software Ecosystems.} 2011.
\item [---] Invited Speaker at {\em PL 2010}, The 3rd Summer School on Programming Languages, Chile. {\em Software Ecosystem Analysis.} 2010.
\item Guest Lecturer at the Faculty of Informatics, University of Lugano. Lecture on {\em Architecture Recovery.} 2010.
\item Guest Lecturer at the Pleiad Lab, University of Chile. Lecture on {\em Reverse Engineering Software Ecosystems.}2010. 
\item Invited Speaker at the 2nd International Workshop on Advanced Software Development Tools and Techniques (Wasdett2 2008), Beijing (China). Invited Talk on {\em Web-based visualization tools for reverse engineering.} 2008.
\item Invited Speaker at IBM TJ Watson Research Center. {\em The Small Project Observatory.} 2007.
\end{itemize}


%%%%%%%%%%%%%%%%%%%%%%%%%%%%%%%%%%%%%%%%%%%%%%%%%%%%%%%%%%%%


%%%%%%%%%%%%%%%%%%%%%%%%%%%%%%%%%%%%%%%%%%%%%%%%%%%%%%%%%%%%
\longcv{
\section{Teaching}
% ==========================================================
\subsection{Courses}
\begin{itemize}
\item Concurrent Programming (2012) (U. Berne)
\item Programming Fundamentals (2004 - 2008) (U. Lugano)
\end{itemize}
} % end longcv


\section{Thesis supervision \period}

\begin{small}
%\subsection{Bachelor and Master Theses supervised or co-supervised}
\begin{enumerate}
\item Jacopo Malnati, X-Ray - An Eclipse Plug-in for Software Visualization. Bachelor Thesis, University of Lugano, 2007.
\item Jacopo Malnati, Developer-centric Analysis of SVN Ecosystems. Master Thesis, University of Lugano, 2009.
\item Alessio Boeckmann, MARS - Modular Architecture Recommendation System. Master Thesis, University of Lugano, 2010
\end{enumerate}
\end{small}
% \ml{should I put supervisions which are "in progress"?}

%%%%%%%%%%%%%%%%%%%%%%%%%%%%%%%%%%%%%%%%%%%%%%%%%%%%%%%%%%%%
\longcv{
\newpage
\section{Publications}

The majority of these publications are available as PDF downloads from the following web page:
\url{http://scg.unibe.ch/staff/mircea/pubs}.

\emph{H-index:} 10.
}

\longcv{
\begin{small}
\input{pubs}
\end{small}
} % end longcv

%%%%%%%%%%%%%%%%%%%%%%%%%%%%%%%%%%%%%%%%%%%%%%%%%%%%%%%%%%%%
\shortcv{
\newpage
\section{Publications \period}

\emph{H-index:} 10.
All publications are available online: \url{http://scg.unibe.ch/staff/mircea/pubs}.

Selected publications particularly relevant to this proposal are indicated by \fbox{$\star$}.
% NB: Generate pubs with "make recent"
\begin{small}


% ==========================================================
\subsection{Peer-reviewed articles (original publications)}

\subsubsection*{Refereed Papers in International Journals}
\begin{enumerate}
\item 
\fbox{$\star$}
Mircea Lungu, Michele Lanza, and Oscar Nierstrasz. \emph{Evolutionary and Collaborative Software Architecture Recovery with Softwarenaut.} In Science of Computer Programming (SCP) p. to appear, 2012.
\item 
Marco D'Ambros, Michele Lanza, Mircea Lungu, and Romain Robbes. \emph{On Porting Software Visualization Tools to the Web.} In In Journal on Software Tools for Technology Transfer 13 p. 181 - 200, 2011.
\item 
\fbox{$\star$}
Mircea Lungu, Michele Lanza, Tudor G\^irba, and Romain Robbes. \emph{The Small Project Observatory: Visualizing Software Ecosystems.} In Science of Computer Programming, Elsevier 75(4) p. 264-275, April 2010.
\item 
Marco D'Ambros, Michele Lanza, and Mircea Lungu. \emph{Visualizing Co-Change Information with the Evolution Radar.} In Transactions on Software Engineering (TSE) 35(5) p. 720 - 735, 2009.
\end{enumerate}


\subsubsection*{Refereed Papers in International Conferences}
\begin{enumerate}
\item 
\fbox{$\star$}
Niko Schwarz, Mircea Lungu, and Romain Robbes. \emph{On How Often is Code Cloned Across Repositories.} In Proceedings of ICSE 2012, p. to appear, 2012. 
\item Erwann Wernli, Mircea Lungu, and Oscar Nierstrasz. \emph{Incremental Dynamic Updates with First-class Contexts.} In Objects, Components, Models and Patterns, Proceedings of TOOLS Europe 2012, 2012. To appear.
\item Amir Aryani, Fabrizio Perin, Mircea Lungu, Abdun Naser Mahmood, and Oscar Nierstrasz. \emph{Can We Predict Dependencies Using Domain information?.} In Proceedings of the 18th Working Conference on Reverse Engineering (WCRE 2011), October 2011. 
\item Lile Hattori and Alberto Bacchelli and Michele Lanza and Mircea Lungu. \emph{Erase and rewind  --  Learning by replaying examples}. In Proceedings of the 24th Conference on Software Engineering Education and Training (CSEET), 2011, Hawaii. 
\item Lile Hattori, Marco D'Ambros, Michele Lanza, and Mircea Lungu. \emph{Software Evolution Comprehension: Replay to the Rescue.} In Proceedings of the 19th International Conference on Program Comprehension, p. 161-170, IEEE Computer Society Press, 2011. 
\item 
\fbox{$\star$}
Romain Robbes and Mircea Lungu. \emph{A Study of Ripple Effects in Software Ecosystems (NIER).} In Proceedings of the 33rd International Conference on Software Engineering (ICSE 2011), p. 904-907, May 2011. 
\item Niko Schwarz, Mircea Lungu, and Oscar Nierstrasz. \emph{Seuss: Cleaning up Class Responsibilities with Language-based Dependency Injection.} In Objects, Components, Models and Patterns, Proceedings of TOOLS Europe 2011, LNCS 33 p. 276-289, Springer-Verlag, 2011. 
\item Toon Verwaest, Camillo Bruni, Mircea Lungu, and Oscar Nierstrasz. \emph{Flexible object layouts: enabling lightweight language extensions by intercepting slot access.} In Proceedings of the 2011 ACM international conference on Object oriented programming systems languages and applications, OOPSLA '11 p. 959-972, ACM, New York, NY, USA, 2011. 
\item 
\fbox{$\star$}
Mircea Lungu, Romain Robbes, and Michele Lanza. \emph{Recovering Inter-Project Dependencies in Software Ecosystems.} In ASE'10: Proceedings of the 25th IEEE/ACM International Conference on Automated Software Engineering, ACM Press, 2010. 
\item Marco D'Ambros, Mircea Lungu, Michele Lanza, and Romain Robbes. \emph{Promises and Perils of Porting Software Visualization Tools to the Web.} In Proceedings of WSE 2009 (11th IEEE International Symposium on Web Systems Evolution), p. 109-118, IEEE CS Press, 2009.
\item 
\fbox{$\star$}
Mircea Lungu and Michele Lanza. \emph{Exploring Inter-Module Relationships in Evolving Software Systems.} In Proceedings of CSMR 2007 (11th European Conference on Software Maintenance and Reengineering), p. 91-100, IEEE Computer Society Press, Los Alamitos CA, 2007.
\item 
\fbox{$\star$}
Mircea Lungu, Michele Lanza, Tudor G\^irba, and Reinout Heeck. \emph{Reverse Engineering Super-Repositories.} In Proceedings of WCRE 2007 (14th Working Conference on Reverse Engineering), p. 120-129, IEEE Computer Society Press, Los Alamitos CA, 2007. 
\item Romain Robbes, Michele Lanza, and Mircea Lungu. \emph{An Approach to Software Evolution Based on Semantic Change.} In Proceedings of FASE 2007 (10th International Conference on Fundamental Approaches to Software Engineering), p. 27-41, 2007.
%\item Marco D'Ambros, Michele Lanza, and Mircea Lungu. \emph{The Evolution Radar: Integrating Fine-grained and Coarse-grained Logical Coupling Information.} In Proceedings of MSR 2006 (3rd International Workshop on Mining Software Repositories), p. 26 - 32, 2006.
%\item Mircea Lungu, Michele Lanza, and Tudor G\^irba. \emph{Package Patterns for Visual Architecture Recovery.} In Proceedings of CSMR 2006 (10th European Conference on Software Maintenance and Reengineering), p. 185-196, IEEE Computer Society Press, Los Alamitos CA, 2006. 
%\item Michael Meyer, Tudor G\^irba, and Mircea Lungu. \emph{Mondrian: An Agile Visualization Framework.} In ACM Symposium on Software Visualization (SoftVis'06), p. 135-144, ACM Press, New York, NY, USA, 2006. 

\end{enumerate}

\subsubsection*{Refereed Papers in International Workshops}
\begin{enumerate}
\item Lile Hattori, Mircea Lungu, and Michele Lanza. \emph{Replaying past changes in multi-developer projects.} In Proceedings of the Joint ERCIM Workshop on Software Evolution (EVOL) and International Workshop on Principles of Software Evolution (IWPSE), p. 13-22, October 2010.
\item Fernando Olivero, Michele Lanza, and Mircea Lungu. \emph{Gaucho: From Integrated Development Environments to Direct Manipulation Environments.} In Proceedings of FlexiTools 2010 (1st International Workshop on Flexible Modeling Tools), 2010.
\item 
\fbox{$\star$}
Mircea Lungu and Tudor G\^irba. \emph{A Small Observatory for Super-Repositories.} In Proceedings of International Workshop on Principles of Software Evolution (IWPSE 2007), p. 106-109, ACM Press, 2007.
%\item Mircea Lungu and Michele Lanza. \emph{Softwarenaut: Cutting Edge Visualization.} In Proceedings of Softvis 2006 (3rd International ACM Symposium on Software Visualization), p. 179-180, ACM Press, 2006.
%\item Mircea Lungu and Michele Lanza. \emph{Softwarenaut: Exploring Hierarchical System Decompositions.} In Proceedings of CSMR 2006 (10th European Conference on Software Maintenance and Reengineering), p. 351-354, IEEE Computer Society Press, Los Alamitos CA, 2006. 
\item Mircea Lungu, Adrian Kuhn, Tudor G\^irba, and Michele Lanza. \emph{Interactive Exploration of Semantic Clusters.} In 3rd International Workshop on Visualizing Software for Understanding and Analysis (VISSOFT 2005), p. 95-100, 2005.

\end{enumerate}

%
\subsection{Monographs}
\begin{enumerate}
\item 
\fbox{$\star$}
Mircea Lungu. \emph{Reverse Engineering Software Ecosystems}. PhD Thesis, University of Lugano, Switzerland. October 2009.
\end{enumerate}
%
% ==========================================================
%\subsection{Book contributions}
%\begin{enumerate}
%\end{enumerate}
%


% ==========================================================
\subsection{Other pertinent publications}
\begin{enumerate}

\item 
\fbox{$\star$}
Mircea Lungu and Oscar Nierstrasz,
``Recovering Software Architecture with Softwarenaut,''
{\it ERCIM News},
vol. 88,
January 2012.

\item 
\fbox{$\star$}
Mircea Lungu, Oscar Nierstrasz and Niko Schwarz
``Big Software Data Analysis''
{\it ERCIM News},
vol. 89,
April 2012 (to appear).

\end{enumerate}
\end{small}

}

% \input{refs}
\end{document}
%%%%%%%%%%%%%%%%%%%%%%%%%%%%%%%%%%%%%%%%%%%%%%%%%%%%%%%%%%%%
