 %\documentstyle[11pt]{article}
\documentclass{article}
\usepackage{color}
\definecolor{darkblue}{rgb}{0,0,0.5}

\usepackage{hyperref}
\hypersetup{ colorlinks, urlcolor=darkblue}

%\setlength{\paperheight}{11in}
\setlength{\textheight}{10in}
\setlength{\topmargin}{-0.5in}
\setlength{\textwidth}{6in}
\setlength{\oddsidemargin}{-0.5in}
%\setlength{\evensidemargin}{0.5in}

%\pagenumbering{plain}

%-------------- define commands and counters ----------------------------------
\renewcommand{\baselinestretch}{1.}
\newcommand{\hs}{\hspace*}
\newcommand{\vs}{\vspace*{2mm}}
\newcommand{\dis}{\displaystyle}
\newcommand{\no}{\noindent}
\newcommand{\ns}{\vspace{6pt}\noindent}
 
%------------------------ my commands ----------------------------------------
\newcommand{\cvsectionname}[1]{\multicolumn{2}{l}{\Large \tt #1}\\\hline\\} 
\newenvironment{cvsection}[1]{\medskip \begin{tabular}{rl} \cvsectionname{#1}}{\end{tabular}}
\newcommand{\cvline}[2]{\parbox[t]{2.3cm}{\sl  \hfill #1} & \parbox[t]{14cm}{ #2 \hfill}\\\vspace{4pt}} 
\newcommand{\cvsecondaryline}[1]{ & \parbox[t]{14cm}{#1 \hfill} \\\vspace{4pt}} 
\newcommand{\cvexplanationline}[1]{ & $\triangleright$ \parbox[t]{14cm}{#1 \hfill} \\\vspace{4pt}} 
\newcommand{\cvtechnicalline}[2]{\parbox[t]{2.3cm}{\sl #1} & \parbox[t]{14cm}{ #2 \hfill}\\\vspace{4pt}} 
\newcommand{\cvlanguageline}[2]{\parbox[t]{2.48cm}{\sl #1} & \parbox[t]{14cm}{ #2 \hfill}\\\vspace{4pt}} 
\newcommand{\cvexperienceline}[2]{\parbox[t]{2.3cm}{\sl \hfill #1} & \parbox[t]{14cm}{{\bf #2} \hfill}\\\vspace{4pt}} 
\newcommand{\cvexperienceexplanationline}[1]{ & \parbox[t]{14cm}{$\triangleright$ #1 \hfill} \\\vspace{4pt}} 
\newcommand{\cvexperiencecontributionline}[1]{ & \parbox[t]{14cm}{\hspace{6pt} $\bullet$ #1 \hfill} \\\vspace{4pt}} 
\newcommand{\cvreferenceline}[2]{& \parbox[t]{16cm}{
%$\bullet$ 
\bf #1, \rm #2}\\\vspace{4pt}} 

\begin{document}

\pagestyle{empty}

\begin{center}
\hspace*{4.8cm}{\LARGE  \bf     Dr. Mircea F. LUNGU}  \\[9pt]
\hspace*{4.8cm}Engestrasse 1 \\
\hspace*{4.8cm}Bern, Switzerland  \\
\hspace*{4.8cm}+41-078-904.3235  \\
\hspace*{4.8cm}mircea.lungu@gmail.com \\
\hspace*{4.8cm}\href{http://scg.unibe.ch/staff/mircea/}{http://scg.unibe.ch/staff/mircea/}
\end{center}

\vspace{0.3cm}

\begin{cvsection}{Education}
\cvline{2009} {PhD degree from the Faculty of Informatics, University of Lugano, Switzerland. 
% My advisor is Prof. Dr. Michele Lanza. My main research interests are in  software engineering and visualization.
}
\cvexplanationline{Advisor is Prof. Michele Lanza}
\cvexplanationline{The PhD thesis proposes techniques and tools for the static and evolutionary analysis of software ecosystems}
% \cvexplanationline{Interested in Software Engineering, Program Comprehension, Software Visualization}

\cvline{2004} {Diploma Engineer degree from ``Politehnica'' University of Timi\c{s}oara}
\cvexplanationline{Average grade for the 5 years: 9.6 (out of 10)}
\cvexplanationline{Diploma thesis focuses on software metrics and visualization}


%\cvline{1995 - 1999}{``Vasile Goldis'' High School in Arad - specialization in informatics}
%\cvexplanationline{Leaving examination mark: 9.5 (out of 10)}
\end{cvsection}



% ~~~~~~~~~~~~~~~~~~~~~~~~~~
\begin{cvsection}{Employment}
% ~~~~~~~~~~~~~~~~~~~~~~~~~~

\cvexperienceline{2010 - present}{Software Composition Group, University of Bern}
\cvexperienceexplanationline{Researcher. Furthering the research on software ecosystem analysis. Collaborating with the PhD students on their various topics. Leading master's and bachelor's projects.}
\cvexperienceexplanationline{Lecturer. Designing the lectures for the {\em Software Design and Evolution} course together with Oscar Nierstrasz. Preparing and grading exams.}

\cvexperienceline{2009 - {2010}}{Faculty of Informatics, University of Lugano, Switzerland}
\cvexperienceexplanationline{Researcher. Conducting research in software engineering, more precisely, architecture recovery and ecosystem analysis. Leading master and bachelor projects.}
\cvexperienceexplanationline{Lecturer. Teaching the {\em Software Atelier II: GUI Design} course. Preparing lectures, grading assignments, preparing projects and exams} 

\cvexperienceline{2004 - 2009}{Faculty of Informatics, University of Lugano, Switzerland}
\cvexperienceexplanationline{PhD Student. Studying techniques and developing tools for architecture recovery and ecosystem analysis. Leading master's and bachelor's projects.}
\cvexperienceexplanationline{Teaching Assistant. Assisting in the {\em Programming Fundamentals I} and co-teaching the {\em Software Atelier II} course. Preparing courses, grading assignments, preparing projects and exams.}



\cvexperienceline{2007, Jul-Dec}{IBM T.J. Watson Research Center, Hawthorne, NY}
\cvexperienceexplanationline {Intern/Co-op in the Next Generation Distributed Systems Group. Extending the Streamsight visualization tool with suport for visualizing and editing semantic descriptions of the components of large distributed workflow applications.}

\cvexperienceline{2001 - 2003}{Computervoice Systems, Romania}
\cvexperienceexplanationline{Software Developer (part-time). Working on Dependabill, a CRM system for  for the telecommunications market in the United States}
%\cvexperiencecontributionline{Team working and learning about industry demands}
%\cvexperiencecontributionline{Implementation in Visual C++}

\cvexperienceline{1999 - 2000}{Logos Highschool, Timi\c{s}oara}
\cvexperienceexplanationline{Instructor. Teaching the introductory programming course with Pascal}
%\cvexperiencecontributionline{Teaching the introductory programming course with Pascal}
\end{cvsection}



% ~~~~~~~~~~~~~~~~~~~~~~~~~~~~~
\begin{cvsection}{Research And Consulting Experience}
% ~~~~~~~~~~~~~~~~~~~~~~~~~~~~~

%\cvexperienceline{Jun. 2011}{Visiting Researcher University California at Irvine}
%\cvexperiencecontributionline{Visiting Researcher.}


\cvexperienceline{Nov-Dec 2010}{Visiting Researcher at the University of Chile}
\cvexperiencecontributionline{Taking part in the research activities of the Pleiad research group}
\cvexperiencecontributionline{Invited speaker at the PL 2010 - Programming Language Summer School}
\cvexperiencecontributionline{Studying ripple effects in software ecosystems in collaboration with Dr. Romain Robbes}

\cvexperienceline{2008}{Consultant for a large telecommunications company}
\cvexperiencecontributionline{Member of a multi-disciplinary team developing an open social networking platform.}
\cvexperiencecontributionline{Member of the team that does rapid prototyping in Ruby.}

\cvexperienceline{2005}{Consultant for a multinational enterprise}
\cvexperiencecontributionline{Member of a team of specialists from across Europe analyzing a legacy software system.}
\cvexperiencecontributionline{Responsible with performing dependency analysis on the system.}

%\cvexperienceline{2004}{Diploma thesis in collaboration with Software Composition Group from University of Bern}
%\cvexperiencecontributionline{Extending the Detection Strategies Mechanism }
%\cvexperiencecontributionline{Developing the Magnet View visualisation tool}
%
\cvexperienceline{2002 - 2004}{Member of the LOOSE Research Group}
\cvexperiencecontributionline{Studying the evolution and re-engineering of object-oriented software systems}

\cvexperiencecontributionline{Reengineering and extending ProDeOOS - a quality assurance tool developed at LRG}
\end{cvsection}

\begin{cvsection}{Community Service}
\cvline{2010}{Lecturer at the PL 2010 - 3rd Summer School on Programming Languages - in Antofagasta, Chile}
\cvline{2005 - 2008}{Organizer of the {\em PhD Talks} at the Faculty of Informatics from University of Lugano. The seminar series are the place where PhD students talk about various topics they are interested in and practice presentation skills.}
\cvline{2005 - present} {Reviewer for conferences and workshops: VisSoft 2005, SoftVis 2006, CSMR 2006, MSR 2006, ICPS 2006, ASE 2006, CSMR 2007, VissSoft 2007, Eurovis 2009, WCRE 2009, CSMR 2010, ICSE 2010, WCRE 2011, Smalltalks 2011.\\
Reviewer for the following Journals: Elsevier SCP Special Edition: EST
Tool Demo Chair for WCRE 2011; Tool Demo Chair for CSMR 2012

}

\end{cvsection}



%~~~~~~~~~~~~~~~~~~~~~~~~~~~~~~
\begin{cvsection}{Distinctions}
%~~~~~~~~~~~~~~~~~~~~~~~~~~~~~~
\cvline{2010}{{Best paper award} with Lile Hattori for the ``Replaying Past Changes in Multi-developer Projects'' article at IWPSE-EVOL 2010}

\cvline{2007}{{\em 1st Place at the ESUG Innovation Awards.} Competing with The Small Project Observatory - an online application aimed ad visualizing project portfolios. ESUG is the primary European Smalltalk Conference.}

\cvline{2006}{{\em Best poster award} for the poster entitled {\em Cutting Edge Software Visualization} presented at the 3rd International ACM Symposium on Software Visualization, Brighton, 2006}
\cvline{2003}{{\em Best software engineer} award in the contest organized by the LOOSE (The Lab on Software Engineering) at the Polytechnic University of Timisoara, Romania}
\cvline{2002}{The 2nd prize with the faculty's team at the ``Hard \& Soft'' International Contest, Suceava, Romania. The contest was based on Image Processing and the project we developed was a security sistem}
\cvline{2001}{The 3rd prize at the Mecrob programming contest, Timisoara, with the software simulation of a mechanical Robot}
\cvline{1999}{The 3rd prize at the National Student Software Development Contest, Focsani, Romania}
%\cvline{1998}{The 1st prize at the Regional Computer Science Contest, Arad}
%\cvline{1995 - 1999}{Prizes at the scholar contests in physics, mathemathics, informatics}
\end{cvsection}



%~~~~~~~~~~~~~~~~~~~~~~~~~~~~~~~~~~~~
\begin{cvsection}{Selected Projects}
%~~~~~~~~~~~~~~~~~~~~~~~~~~~~~~~~~~~~
\cvline{2007 - ... } {\href{http://spo.inf.usi.ch/}{\bf The Small Project Observatory} - Framework for software ecosystem visualization and analysis. The project won the First Place in the ESUG Innovation Award Competiotion in 2007 (ESUG - European Smalltalk Conference)}

\cvline{2006}  {\href{http://www.inf.usi.ch/phd/lungu/iretrospect/}{\bf iRetrospect}. A pet project that I started in order to be able to visualise the way I spend my time on the computer. In the same time I wanted to experience development for OS X with Cocoa and ObjectiveC.}
\cvline{2005 - ...}{\href{http://www.inf.usi.ch/phd/lungu/softwarenaut/}{\bf Softwarenaut}. A tool for software visualization and exploration. The goal of the tool is understanding large software systems by exploring their hierarchical decompositions.  The application is developed using VisualWorks Smalltalk}
\cvline{2004}{MagnetView. A software visualization tool for visualizing software artefacts and metrics}
%\cvline{2003}{Kitta. A kindergarten management application written as a team project during the LOOSE Software Engineering Laboratory. ``Best Software engineer'' award}
\cvline{2002}{Mircompilatorul. A compiler for Pascal written in Java}
\cvline{2000}{Double-Triple-R. A 3D simulation of a robot. 3rd place at the robotics contest}
\end{cvsection}




% ~~~~~~~~~~~~~~~~~~~~~~
\begin{cvsection}{Selected Publications}
% ~~~~~~~~~~~~~~~~~~~~~~

\cvline {2011} { { \bf A Study of Ripple Effects in Software Ecosystems.} Romain Robbes and Mircea Lungu. In Proceedings of the 33rd International Conference on Software Engineering (ICSE 2011), p. 904—907, May 2011.}

\cvline {} { {\bf Can We Predict Dependencies Using Domain information?} Amir Aryani, Fabrizio Perin, Mircea Lungu, Abdun Naser Mahmood, and Oscar Nierstrasz. In Proceedings of the 18th Working Conference on Reverse Engineering (WCRE 2011), October 2011. }

\cvline {2010} { { \bf The Small Project Observatory: Visualizing Software Ecosystems.}
Mircea Lungu, Michele Lanza, Tudor Girba, Romain Robbes
In Journal of Science of Computer Programming (SCP), Vol. 75, No. 4, pp. 264 - 275. Elsevier, 2010.
}

%\cvline {} {{\bf Gaucho: From Integrated Development Environments to Direct Manipulation Environments.}
%Fernando Olivero, Michele Lanza, Mircea Lungu
%In Proceedings of FlexiTools 2010 (1st International Workshop on Flexible Modeling Tools).
%}
%
\cvline {} {{\bf The Small Project Observatory - A Tool for Reverse Engineering Software Ecosystems.}
Mircea Lungu, Michele Lanza
In Proceedings of ICSE 2010 (32nd ACM/IEEE International Conference on Software Engineering), pp. 289 - 292, ACM Press, 2010.
}


%\cvline {} {{\bf Promises and Perils of Porting Software Visualization Tools to the Web}
%Marco D'Ambros, Mircea Lungu, Michele Lanza, Romain Robbes
%In Proceedings of WSE 2009 (11th IEEE International Symposium on Web Systems Evolution), pp. 109 - 118. IEEE CS Press, 2009.
%}
%
\cvline {2008} {{\bf Towards reverse engineering software ecosystems.}
In proceedings of ICSM 2008 (24 IEEE Conference on Software Maintenance), 
pp. 428-431, IEEE Press, 2008. Doctoral Symposium. }

\cvline {} {{\bf A Teamwork-Based Approach to Programming Fundamentals with Scheme, Smalltalk, and Java}
Michele Lanza, Amy Murphy, Romain Robbes, Mircea Lungu, Paolo Bonzini, Marco D'Ambros, Richard Wettel
In Proceedings of ICSE 2008 (30th International Conference on Software Engineering, Education Track), pp. 787 - 790, ACM Press, 2008.
}


\cvline{2007} {{\bf Reverse Engineering Super-Repositories.}
Mircea Lungu, Michele Lanza, Tudor Girba, Reinout Heeck
In Proceedings of WCRE 2007 (14 Conference on Reverse Engineering), pp. 120 - 129, IEEE Computer Society, 2007.}

\cvline{} {{\bf Exploring Inter-Module Relationships in Evolving Software Systems}, Mircea Lungu, Michele Lanza. In Proceedings of CSMR 2007 (11th European Conference on Software Maintenance and Reengineering), pp. 91- 100, IEEE Computer Society, 2007}


%\cvline{2006} {{\bf Package Patterns for Visual Archietcure Recovery}. Mircea Lungu, Michele Lanza, Tudor Girba. In Proceedings of CSMR 2006 (10th European Conference on Software Maintenance and Reengineering), pp. 183 - 192, IEEE Computer Society, 2006}

%\cvline{} {{\bf Softwarenaut: Exploring Hierarchical System Decompositions} (Tool Demonstration) Mircea Lungu, Michele Lanza. In Proceedings of CSMR 2006 (10th European Conference on Software Maintenance and Reengineering), pp. 349 - 350, IEEE Computer Society, 2006}
%
%\cvline{} { {\bf Softwarenaut: Cutting the Edge in Software Visualization}.  (Best Poster Award)
%Mircea Lungu, Michele Lanza. In Proceedings of the 3rd International ACM Symposium on Software Visualization (SoftVis 2006), pp. 179-180}

\cvline{} {{\bf Mondrian: An Agile Information Visualization Framework}. Michael Meyer, Mircea Lungu and Tudor Girba. In Proceedings of the 3rd International ACM Symposium on Software Visualization, (SoftVis 2006), pp. 135-144.}
\cvline{} {{\bf The Evolution Radar: Visualizing Integrated Logical Coupling Information}.   
Marco D'Ambros, Michele Lanza, Mircea Lungu. In Proceedings of MSR 2006 (3rd International Workshop on Mining Software Repositories), pp. 26-32, 2006.}

\cvline{2005} {{\bf Interactive Exploration of Semantic Clusters}. Mircea Lungu, Adrian Kuhn, Tudor Girba, Michele Lanza. In Proceedings of VISSOFT 2005 (3rd IEEE International Workshop on Visualizing Software For Understanding and Analysis), pp. 95 - 100, IEEE CS Press, 2005. }

\end{cvsection}

\begin{cvsection}{Books, Book Chapters and Theses}

\cvline {2009} {{\bf Reverse Engineering Software Ecosystems.}
Mircea Lungu
PhD Thesis, University of Lugano, Switzerland. October 2009.
}

\cvline{2006} {{\bf Biomedical Information Visualization}, Mircea Lungu, Kai Xu. Chapter in {\em Human Centered Visualization Environments}. Springer LNCS (to appear).}

\cvline{2004} {{\bf Conformity Strategies: Measures of Software Design Rules }. Diploma thesis defended at the "Politehnica" University of Timisoara (Romania), Department of Computer Science }
\end{cvsection}


% ~~~~~~~~~~~~~~~~~~~~~~
%\begin{cvsection}{Presentations}
%% ~~~~~~~~~~~~~~~~~~~~~~
%\cvline{2007} {{\bf Exploring Inter-Module Relationships in Evolving Software Systems}. CSMR, Amsterdam 2007}
%
%\cvline{2006} {{\bf A Survey of Biomedical Visualization}. HCVE, Dagstuhl 2006}
%\cvline{} {{\bf Softwarenaut: Cutting Edge Software Visualization}. SoftVis, Brighton 2006}
%\cvline{} {{\bf Package Patterns for Visual Architecture Recovery}. CSMR, Bari 2006}
%
%\cvline{2005} {{\bf Interactive Exploration of Semantic Clusters}. IEEE VISSOFT, Budapest 2005}
%\cvline{2003} {{\bf Optimizing Prodeoos, a tool for detecting design flaws}. SYNASC, Timisoara, 2003}
%\end{cvsection}
%
%\begin{cvsection}{Technical Skills}
%\cvtechnicalline{Languages:}{C/C++, Java, Smalltalk, Lisp, Prolog, Pascal,  Assembler, SQL, Php}
%\cvtechnicalline{OS:}{Windows, Unix}
%\cvtechnicalline{Tools:}{Visual Studio, Access, SQLServer, JBuilder, Delphi, Emacs, Idea, VisualWorks, Matlab}
%\cvtechnicalline{Others:}{EJB, JSP, Ant, XUnit, MySQL, MFC, Flex, Bison, DirectX, SDL}
%\end{cvsection}




\begin{cvsection}{Programming Languages} 
\cvlanguageline{Lisp} {I used to teach {\em Scheme} at undergraduate level and enjoy writing Lisp for small projects} %(e.g., currently I am writing a musical counterpoint generator).}

\cvlanguageline{Smalltalk} {For the last three years I have been working with {\em Smalltalk} and I came to appreciate the flexibility of the language and it's simplicity. At the university of lugano I taught smalltalk during the Programming Fundamentals course.}

\cvlanguageline{Java} {During the university years, I wrote a compiler for a simplified version of Pascal and a simplle CRM application. During my 6 months of internship at IBM I developed an Eclipse plugin for visualizing large distributed systems specifications. I supervised a bachelor project implementing a software visualization plugin for the Eclipse IDE.}

\cvlanguageline{Other} {While working as a software developer at Computervoice Systems, I acquired in-depth knowledge of  {\em Visual C++} and MFC. I have taught {\em Pascal} at highschool level. Over the years I worked with many other languages: as Assembly, Prolog, Objective C, Ruby.}
\end{cvsection}



% ~~~~~~~~~~~~~~~~~~~~~
\begin{cvsection}{Languages} 
% ~~~~~~~~~~~~~~~~~~~~~
\cvlanguageline{\hspace{1cm}}{{\em Romanian} is my native language. I speak fluently {\em English} and I have good knowledge of {\em Italian} and {\em French}. I have basic German conversation skills.}
\end{cvsection}
\vspace{1cm}

%% ~~~~~~~~~~~~~~~~~~~~~
%\begin{cvsection}{Other} 
%% ~~~~~~~~~~~~~~~~~~~~~
%\cvlanguageline{\hspace{1cm}}{
%I like to play the guitar and hike. I was the organizer of the {\em Lugano Funshops}, an improvisational theatre workshop for the students of the University of Lugano.
%}
%\end{cvsection}
%\vspace{1cm}
%

% ~~~~~~~~~~~~~~~~~~~~~
\begin{cvsection}{References}
% ~~~~~~~~~~~~~~~~~~~~~

\cvreferenceline{Prof. Dr. Oscar Nierstrasz}{\\Software Composition Group, University of Bern. 
\\Email: oscar@iam.unibe.ch\\}


\cvreferenceline{Prof. Dr. Michele Lanza}{\\Faculty of Informatics, University of Lugano. 
\\Email: michele.lanza@usi.ch\\}

\cvreferenceline{Prof. Dr. Radu Marinescu}{\\Faculty of Automation and Computer Science\\ ``Politehnica'' University of Timi\c{s}oara. 
\\Email: radum@cs.utt.ro\\}

\cvreferenceline{Dr. Wim de Pauw}{\\IBM T.J. Watson Research Center. 
\\Email: wim@us.ibm.ch\\}

\cvreferenceline{Dipl. Ing. Calin Sircuta}{\\Computervoice Systems, Romania. 
\\Email: sircux@computervoice.ro\\}
	
\end{cvsection}


\end{document}



